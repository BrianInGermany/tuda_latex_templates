\documentclass[
	ngerman,
	accentcolor=9c,% Farbe für Hervorhebungen auf Basis der Deklarationen in den 
	type=intern,
	marginpar=false
	]{tudapub}
	
\usepackage[english, main=ngerman]{babel}
\usepackage[babel]{csquotes}

%Formatierungen für Beispiele in diesem Dokument. Im Allgemeinen nicht notwendig!
\let\file\texttt
\let\code\texttt
\let\pck\textsf
\let\cls\textsf

\begin{document}
	
	
\title{\texorpdfstring{TUDaPub -- \LaTeX-Paper im Corporate Design der TU Darmstadt}{TUDaPub -- LaTeX-Paper im Corporate Design der TU Darmstadt}}
\subtitle{Die Dokumentenklasse tudapub}
\author{Marei Peischl}

\maketitle

\tableofcontents

\section{Über diese Datei}
Die Datei \file{DEMO-TUDaPub.tex} beziehungsweise ihre Ausgabe \file{DEMO-TUDaPub.pdf} ist die Dokumentation der Dokumentenklasse \file{tudapub.sty}.

Sie ist Teil des TUDa-CI-Bundles und basiert in Teilen auf dem tuddesign-Paket von C.~v.~Loewenich und J.~Werner. 
	
In diesem Dokument werden die speziellen Optionen und Einstellungsmöglichkeiten erläutert.
\end{document}
