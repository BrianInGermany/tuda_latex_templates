\documentclass[
	ngerman,%globale Übergabe der Hauptsprache
	aspectratio=169,%Beamer eigene Option zum Umschalten des Formates
	color={accentcolor=2d},
	logo=false,%Kein Logo auf Folgeseiten
	colorframetitle=true,%Akzentfarbe auch im Frametitle
%	logofile=example-image, %Falls die Logo Dateien nicht vorliegen
	]{tudabeamer}
\usepackage[main=ngerman]{babel}

% Der folgende Block ist nur bei pdfTeX auf Versionen vor April 2018 notwendig
\usepackage{iftex}
\ifPDFTeX
\usepackage[utf8]{inputenc}%kompatibilität mit TeX Versionen vor April 2018
\fi


%Makros für Formatierungen der Doku
%Im Allgemeinen nicht notwendig!
\let\code\texttt

\begin{document}

\title{LaTeX-Beamer im Corporate Design der TU Darmstadt}
\subtitle{Version 1.0}
\author[M. Peischl]{Marei Peischl \and der \TeX-Löwe}
\department{\TeX/\LaTeX}
\institute{pei\TeX}

%Fremdlogo
%Logo macro mit sternchen skaliert automatisch, sodass das logo in die Fußzeile passt
\logo*{\includegraphics{example-image-16x9}}

% Da das Bild frei wählbar nach Breite und/oder Höhe skaliert werden kann, werden \width/\height entsprechend gesetzt. So kann die Fläche optimal gefüllt werden.
%Sternchenversion skaliert automatisch und beschneidet das Bild, um die Fläche zu füllen.
\titlegraphic*{\includegraphics{example-image}}


\date{15. Mai 2019}

\maketitle


\section{Dokumentation}
\begin{frame}{Die Dokumentenklasse tudabeamer}
\begin{itemize}
	\item Verwendung wie beamer
	\item keine besondere Syntax notwendig
	\item Klassenoption accentcolor wählt Akzentfarbe
	\item Option serif=true für Serifenschrift
\end{itemize}
\end{frame}

\begin{frame}{Zusätzliche Features der Titelfolie}
\begin{itemize}
	\item \code{\textbackslash{}logo} wählt Fremdlogo für Fußzeile
	\item \code{\textbackslash{}titlegraphic} Ersetzt den unteren Teil der Titelfolie. Zusätzlich existiert \code{\textbackslash{}titlegraphic*{Inhalt}}.
	In diesem Fall wird der Inhalt in eine Box gesetzt, die so skaliert wird, dass Sie den Bereich des Titelbildes überdeckt und entsprechend mittig ausgeschnitten.
\end{itemize}
\end{frame}

\setupTUDaFrame{logo=true}
\begin{frame}[fragile]{Logo im Frametitle}
Das Logo innerhalb des Frametitle kann mit der Klassenoption \code{logo=false} ausgeschalten werden.

Soll das Logo später für ein Folie oder einen Bereich wieder aktiviert werden, steht das Makro
\begin{verbatim}
\setupTUDaFrame{logo=true}
\end{verbatim}
Zur Verfügung. Dort kann die globale Einstellung lokal überschrieben werden.
\end{frame}

\begin{frame}{Frame mit Untertitel}
\framesubtitle{Untertitel}
Ein Beispiel.
\end{frame}

\begin{frame}{Blöcke}
\begin{block}{Standardblock mit Titel}
	Blockinhalt
\end{block}
\begin{block}{}
	Ohne Titel
\end{block}
\end{frame}

\begin{frame}{Spezielle Blöcke}
\begin{exampleblock}{Exampleblock}
	Blockinhalt
\end{exampleblock}
\begin{alertblock}{Alertblock}
	Blockinhalt
\end{alertblock}
\begin{example}[Für die example-Umgebung]
	Inhalt
\end{example}
\end{frame}


\begin{frame}[fragile]{Anpassungen der Mathematikschriftarten}
	Es gibt keine fest Vorgabe zur Verwendung einer Mathematikschrift.
	
	In der Diskussion (\url{https://github.com/tudace/tuda_latex_templates/issues/30}) hat sich folgendes als hinreiche Lösung herausgestellt. Jedoch funktioniert diese Lösung nicht in pdflatex!
	\begin{verbatim}
	\usepackage{unicode-math}
	\setmathfont{Fira Math}
	\setmathfont[range=up]{Roboto}
	\setmathfont[range=it]{Roboto-Italic}
	\setmathfont[range=\int]{Fira Math}
	\end{verbatim}
	Allgemein kann jedoch die Mathematikschriftart wie	auch sonst durch Pakete angepasst werden.
\end{frame}
\end{document}

